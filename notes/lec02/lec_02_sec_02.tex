% !TEX root=./../maha-dip-notes.tex
\section{Histogram Equalization}

Histogram equalization is a method to adjust image intensities to enhance contrast. The histogram of an image is "flattened" (equalized), redistributing pixel values as uniformly as possible over the intensity range.

\dfn{Histogram Equalization}{A point-wise image transformation that remaps the intensity distribution so that the output histogram is (approximately) uniform. This increases the dynamic range and enhances global contrast.}

\noindent Given a gray scale image with intensities $r$ in $[0, K-1]$ and histogram $h(r)$, histogram equalization computes a transformation $HE$ as the cumulative distribution function (CDF):
$$
HE(r) = \sum_{k=0}^{r} \frac{h(k)}{MN}
$$
where $MN$ is the total number of pixels.

\dfn{Histogram Equalization Transformation}{
The transformation $s = HE(r)$ maps each input intensity $r$ to an output intensity $s$ so that $s$ is approximately uniformly distributed over $[0, K-1]$.
}

\paragraph{Step-by-step Procedure}
\begin{enumerate}
    \item Calculate the histogram $h(r)$ of image intensities.
    \item Normalize to get probability distribution $p(r)$.
    \item Compute cumulative distribution $HE(r)$.
    \item Map each $r$ by $s = (K-1) \cdot HE(r)$.
    \item Replace each $f(x, y)$ with $s$.
\end{enumerate}

\ex{Histogram Equalization Examp    le}{
Consider a $4 \times 4$ image:
$$
\begin{bmatrix}
52 & 55 & 61 & 59 \\
79 & 61 & 76 & 61 \\
85 & 70 & 79 & 55 \\
52 & 55 & 61 & 59
\end{bmatrix}
$$
Calculate $h(r)$, $T(r)$, and remap all values using $s = (L-1)T(r)$. This leads to expanded dynamic range and visually improved contrast.
}

\paragraph{Applications of Histogram Equalization}
\begin{itemize}
    \item Enhancement of images with poor contrast.
    \item Preprocessing for computer vision tasks.
    \item Medical imaging, remote sensing.
\end{itemize}
\nt{Histogram equalization may produce undesirable artifacts if the histogram contains large peaks or is multimodal; local methods may better preserve detail.}

\subsection{Adaptive Histogram Equalization (AHE)}

\dfn{Adaptive Histogram Equalization (AHE)}{A method where histogram equalization is performed on small local regions of the image rather than globally, better enhancing local contrast.}

\paragraph{Use and Algorithm}
AHE divides the image into tiles, applies histogram equalization to each, and then smoothly stitches them. This improves visibility of features in small regions.

\clm{Local Contrast Advantage}{}{
AHE enhances the contrast of features that may be obscured by global equalization, especially in images with spatially varying lighting.
}
\paragraph{Derivation}
Let $W_{i,j}$ denote the window centered at pixel $(i, j)$. For each window, compute its local histogram and derive $T_{i, j}(r)$, apply mapping to pixels at $(i, j)$.

\nt{AHE can amplify noise in homogeneous areas; improved methods like Contrast Limited AHE (CLAHE) restrict histogram peaks to reduce artifacts.}

\subsection{Histogram Matching (Specification)}

\dfn{Histogram Matching}{A process to transform an image’s histogram to resemble a specified target histogram—useful for standardizing brightness and contrast between images.}

\paragraph{Use}
Used in image normalization, look adaptation in film, remote sensing for compensating illumination differences.

\paragraph{Derivation}
Let $T(r)$ be the equalization transform for the input image, and $G(z)$ for the target histogram:
\begin{itemize}
    \item For each pixel with intensity $r$, find $s = T(r)$.
    \item Find $z$ such that $G(z) \approx s$.
    \item Map $f(x, y)$ to $z$.
\end{itemize}

\clm{Histogram Matching Algorithm}{}{
Applying histogram equalization to both source and reference image, then mapping levels by the inverse CDF, produces an output whose histogram matches the reference.
}
\ex{Histogram Matching Example}{
Suppose $T(r)$ maps $[0, 255]$ to $[0.2, 0.6]$ and $G(z)$ maps its CDF to $[0, 1]$. For $f(x, y) = 61$, $s = T(61) = 0.3$, $z = G^{-1}(0.3)$, so output pixel is set to $z$.
}

\nt{Histogram matching is essential in applications where uniformity across image datasets is required, such as medical or satellite image analysis.}

\section{Additional Theories and Extensions}

\paragraph{Retinex Theory}
Retinex posits that the perceived color of a pixel ($I$) is the product of the light source ($L$) and the reflectance ($R$):
$$
I(x, y) = L(x, y) \cdot R(x, y)
$$
Retinex-based algorithms attempt to decompose an image into illumination and reflectance components for improved color constancy and dynamic range.

\paragraph{Dark Channel Prior}
Used in haze removal, the \emph{dark channel prior} exploits the property that in most non-sky patches of outdoor images, at least one color channel has very low intensity:
$$
\text{dark}(x, y) = \min_{c \in \{r,g,b\}} \left( \min_{(i, j) \in \Omega(x, y)} I^c(i,j) \right)
$$
where $\Omega$ is a local patch. This prior helps estimate atmospheric light and transmission map for dehazing.

\nt{Retinex and dark channel prior are advanced theories, bridging classical image processing and physics-based models for real-world improvements.}

\section{Deep Learning Approaches}

Recent methods leverage deep learning to adapt intensity transformations, contrast normalization, or even build custom histogram-equalization-like layers in neural networks\cite{szeliski2010,govindjee2007,bovik2019,gonzalez2008}.

\clm{Learned Point Operations}{}{
Deep networks can be trained to perform sophisticated global and local contrast adjustments based on task-specific criteria, outperforming traditional fixed algorithms in complex scenarios.
}


